% ============================================================
% CVUT Starting Grant — Scientific Project Description (Template)
% NOTE: Headings must match Annex B exactly.
%
% Policy implemented here:
%   - HARD LIMIT: 6 pages TOTAL for the Scientific Project Description
%     (includes all text, figures, tables, captions, equations, footnotes, and in-text citations).
%   - HARD LIMIT: up to 2 additional pages for REFERENCES ONLY (references only; no narrative text).
%   - Section "½ page / 1 page / 2 pages" are TARGETS, not hard caps.
%
% Quick start:
%   - Overleaf: upload as main.tex and compile with pdfLaTeX
%   - Local:    latexmk -pdf -interaction=nonstopmode main.tex
%
% Draft mode:  guidance notes, target lengths, and assessment-criteria prompts are visible.
% Final mode:  set \guidancefalse and DELETE all \lipsum text.
% ============================================================

\documentclass[11pt,a4paper]{article}

% --- Page layout (minimum required) ---
\usepackage[a4paper,margin=2cm]{geometry}
\usepackage{setspace}
\setstretch{1.0} % single (or greater)

% --- Times-based font ---
\usepackage{newtxtext}
\usepackage{newtxmath}

% --- Useful basics ---
\usepackage{microtype}
\usepackage{graphicx}
\usepackage{booktabs}
\usepackage[hidelinks]{hyperref}

% --- Tidy lists ---
\usepackage{enumitem}
\setlist{nosep}

% --- Header/footer (simple page number for applicant self-check) ---
\usepackage{fancyhdr}
\pagestyle{fancy}
\fancyhf{}
\lhead{Applicant's last name}
\chead{Scientific project description}
\rhead{Project Acronym}  
\cfoot{\thepage}

% ============================================================
% Guidance toggle (recommended)
%   - Keep \guidancetrue while drafting.
%   - Switch to \guidancefalse for the final PDF (hides notes/prompts).
% ============================================================
\newif\ifguidance
\guidancetrue
% \guidancefalse

% --- Helper: HARD LIMIT box (draft mode only) ---
\newcommand{\HardLimitBox}{%
\ifguidance
\vspace*{-0.25\baselineskip}
\begin{center}
\fbox{%
\begin{minipage}{0.96\linewidth}
\small
\textbf{Hard limits (submission rules):}
\begin{itemize}
  \item \textbf{Scientific Project Description: max 6 pages total} (from the first required heading to the last),
  including \textbf{all} text, figures, tables, captions, equations, footnotes, and in-text citations.
  \item \textbf{References: up to 2 additional pages, references only.} No narrative text, appendices, or extra sections.
  \item Use the \textbf{exact headings} below (no renaming). Content order should follow the headings.
  \item Formatting: \textbf{A4, \(\ge\)11 pt, margins \(\ge\)2 cm, single spacing or greater}.
  \item Practical tip: Draft with \texttt{\textbackslash guidancetrue}, finalize with \texttt{\textbackslash guidancefalse}.
\end{itemize}
\textbf{Section targets:} The “½ / 1 / 2 page” guidance below is \textbf{indicative only}
and is \textbf{not} used for formal compliance checks.
\end{minipage}}
\end{center}
\vspace*{0.5\baselineskip}
\fi
}

% --- Helper: per-section TARGET note (draft mode only) ---
% #1 = target pages; #2 = optional rough word-count guidance
\newcommand{\TargetNote}[2]{%
\ifguidance
\noindent\textit{Target length (indicative): #1%
\if\relax\detokenize{#2}\relax\else\space\textemdash\space #2\fi.}\par\vspace{0.25\baselineskip}
\fi
}

% --- Helper: assessment criteria prompt (draft mode only) ---
\newcommand{\AssessmentCriteria}[1]{%
\ifguidance
{\color{gray}
\vspace{-0.25\baselineskip}
\begin{quote}
\small
\textbf{Assessment criteria (referees will consider):}
\begin{itemize}[leftmargin=*]
#1
\end{itemize}
\end{quote}}
\fi
}

% --- Optional helper: a compact subheading for within-section structure ---
\newcommand{\Subhead}[1]{\vspace{0.25\baselineskip}\noindent\textbf{#1}\par}

\begin{document}

\HardLimitBox

% ============================================================
% REQUIRED HEADINGS (use exactly these; content below is filler)
% ============================================================

\section*{Extended Synopsis}
\TargetNote{about ½ page}{roughly 250--350 words}
\AssessmentCriteria{
  \item Clarity of the central idea and its novelty.
  \item Significance and expected scientific/technological impact.
  \item Coherence of objectives, approach, and expected outcomes.
  \item Overall readability and conciseness for non-specialists in the broader field.
}

\section*{State of the Art and Gap}
\TargetNote{about 1 page}{roughly 500--700 words}
\AssessmentCriteria{
  \item Demonstrated understanding of the international state of the art.
  \item Clear identification of the key gap/limitation and why it matters now.
  \item Appropriate positioning versus the most relevant competing approaches/results.
  \item Use of key references (quality over quantity).
}

\section*{Objectives and Originality}
\TargetNote{about ½ page}{roughly 250--350 words}
\AssessmentCriteria{
  \item Objectives are clear, focused, and realistically achievable within the project duration.
  \item Originality and ambition (incl. high-risk/high-gain elements where appropriate).
  \item Expected contribution to knowledge and/or enabling technology.
  \item Internal consistency: objectives follow logically from the identified gap.
}

\section*{Approach and Work Plan}
\TargetNote{about 2 pages}{roughly 1000--1400 words (depends on figures/tables)}
\AssessmentCriteria{
  \item Appropriateness and rigor of the proposed methodology.
  \item Quality of the work plan (logic, work packages/tasks, milestones, deliverables).
  \item Evidence of feasibility (preliminary results, access to methods/data, partner inputs).
  \item Alignment between objectives, methods, and resources (people/equipment/time).
  \item Where relevant: plans for data, software, and research reproducibility.
}
% Optional internal structure (delete if not needed)
\ifguidance
\Subhead{Suggested structure (optional)}
\begin{itemize}[leftmargin=*]
  \item Work packages / tasks (WP1--WPn) with clear outputs
  \item Milestones and decision points
  \item Optional: a small Gantt-style figure (counts toward the 6-page limit)
\end{itemize}
\fi

\section*{Feasibility, Risks, and Mitigation}
\TargetNote{about 1 page}{roughly 500--700 words}
\AssessmentCriteria{
  \item Key risks are identified (scientific/technical/operational) and prioritised.
  \item Mitigation strategies are credible (fallback options, alternative methods, stop/go criteria).
  \item \textbf{Timeline realism:} critical-path dependencies are understood and buffered.
  \item \textbf{Timing risks and contingency:} hiring, procurement, approvals/ethics, data access, facility availability.
  \item Overall plausibility that objectives can be delivered within the project duration.
}
% Optional internal structure (recommended; concise)
\ifguidance
\Subhead{Suggested structure (recommended)}
\begin{itemize}[leftmargin=*]
  \item Top 3--5 risks (what can go wrong)
  \item Mitigation (what you will do)
  \item \textbf{Timing risks \& contingency plan} (critical path + what you will do if delayed)
\end{itemize}
\fi

\section*{Fit with CTU and Research Group Setup}
\TargetNote{about 1 page}{roughly 500--700 words}
\AssessmentCriteria{
  \item Credibility of the PI’s plan to establish an independent research group at CTU.
  \item Quality of the recruitment and team-building plan (roles, timing, supervision, openness of search).
  \item Fit with CTU: clarity of the group’s niche and how it complements existing strengths.
  \item Integration plan: access to infrastructure/support and collaborations/mentoring that support independence and growth.
  \item Sustainability beyond the grant (follow-up funding strategy and long-term embedding); the plan should remain robust under host matching after shortlisting.
}
\ifguidance
\Subhead{Suggested structure (optional)}
\begin{itemize}[leftmargin=*]
  \item Group vision and niche at CTU (2--3 sentences)
  \item Team and recruitment plan (roles, indicative timing, supervision; open search approach)
  \item Independence and integration (mentoring/advisory arrangements; internal/external collaborations; access to key infrastructure)
  \item Sustainability (follow-up funding plan; contribution to CTU research and training)
  \item Optional: preferred host unit(s) (non-binding); avoid repeating formal host commitments provided in the host unit statement
\end{itemize}
\fi

% ============================================================
% REFERENCES (include only if your workflow expects references appended
% in the same PDF; otherwise submit references as a separate PDF)
% ============================================================

\clearpage
\section*{References}
\ifguidance
\noindent\textit{Hard limit: up to 2 additional pages. References only (no narrative text).}\par
\vspace{0.25\baselineskip}
\fi

% Option A: manual references
\begin{thebibliography}{99}
\bibitem{key1} Author(s). Title. \emph{Journal/Publisher}, Year.
\bibitem{key2} Author(s). Title. \emph{Journal/Publisher}, Year.
\end{thebibliography}

% Option B: BibTeX (uncomment if you prefer BibTeX)
% \bibliographystyle{plain}
% \bibliography{references}

\end{document}